%	Slutrapportmall för examensarbete i LaTeX
% Av: Olle Calderon 2017
% Krav: TeXLive samt en TeX-editor. Jag rekommenderar editorn TeXmaker: http://www.xm1math.net/texmaker/
% Användning: Placera denna fil ("header.tex") i samma mapp som din rapports TeX-fil ("valfritt_filnamn.tex"). I din rapports TeX-fil, skriv följande rad först: %	Slutrapportmall för examensarbete i LaTeX
% Av: Olle Calderon 2017
% Krav: TeXLive samt en TeX-editor. Jag rekommenderar editorn TeXmaker: http://www.xm1math.net/texmaker/
% Användning: Placera denna fil ("header.tex") i samma mapp som din rapports TeX-fil ("valfritt_filnamn.tex"). I din rapports TeX-fil, skriv följande rad först: %	Slutrapportmall för examensarbete i LaTeX
% Av: Olle Calderon 2017
% Krav: TeXLive samt en TeX-editor. Jag rekommenderar editorn TeXmaker: http://www.xm1math.net/texmaker/
% Användning: Placera denna fil ("header.tex") i samma mapp som din rapports TeX-fil ("valfritt_filnamn.tex"). I din rapports TeX-fil, skriv följande rad först: %	Slutrapportmall för examensarbete i LaTeX
% Av: Olle Calderon 2017
% Krav: TeXLive samt en TeX-editor. Jag rekommenderar editorn TeXmaker: http://www.xm1math.net/texmaker/
% Användning: Placera denna fil ("header.tex") i samma mapp som din rapports TeX-fil ("valfritt_filnamn.tex"). I din rapports TeX-fil, skriv följande rad först: \input{header}
% Detta inkluderar koden från denna fil (header.tex).

\documentclass[11pt,a4paper,twoside]{article}

%----------------- PACKAGES -----------------
\usepackage[swedish]{babel}			% Avstavning, språk etc: SWÄRJE
\usepackage[hidelinks]{hyperref}	% Clickable references, links etc.
\usepackage[left=32mm, right=32mm, top=32mm, bottom=32mm]{geometry}		% Page margins
\usepackage[utf8]{inputenc}	% UTF-8 encoding (good for e.g. åäö) 
\usepackage[T1]{fontenc}	% T1 encoding
\usepackage{graphicx}	% For figures
\usepackage{wrapfig}	% For text-wrapped figures
\usepackage{titling}	% For title info etc.
\usepackage{titlesec}	% Formatting of sections etc.
\usepackage{helvet}		% Helvetica, Arial:ish
\usepackage{tocloft}	% ToC fixes
\usepackage{fancyhdr}	% For header and footer formatting
\usepackage{tikz}		% Life sucks without TikZ
\usepackage{circuitikz}	% Life sucks even harder without circuitz
\usetikzlibrary{backgrounds}	% For doing tikz on top of pics
\usepackage{pgfplots}	% For doing plots and stuff
\usepackage{pdfpages}	% For importing cover pages
\usepackage{amsmath}	% Mathematical symbols
\usepackage{gensymb}	% More math symbols
\usepackage[font={footnotesize}]{subcaption}	% For subfigures
\usepackage[font={footnotesize}]{caption}		% Reduce caption font size
\usepackage{float}		% For fancy float options
\usepackage{graphics}	% For placing graphics in /pics/ folder
\usepackage{xcolor}		% Fancy colors
\usepackage{IEEEtrantools}	% For IEEE style formatting
\usepackage{verbatim} % For comment macro
\usepackage{microtype} % Squeeze text to fit margins
\usepackage{listings}	% Listings for code snippets etc.
\usepackage{xcolor}	% Color package
\usepackage{color}	% More colors
\usepackage{setspace}	% Adjust spacings
\usepackage{datetime} % this month etc

%\author{\firstauthorname\\ \secondauthorname}	

%----------------- SETTINGS -----------------
\setlength\parindent{0pt}	% No paragraph indents.
\graphicspath{{./pics/}}		% Put figures in /pics/ folder

\raggedbottom	% Solves problem where section spacing is wacky
\setstretch{1.0}

\bibliographystyle{IEEEtran}


%----------------- C-kod -----------------
\lstdefinestyle{customc}{
  belowcaptionskip=1\baselineskip,
  breaklines=true,
  captionpos=b,
  frame=trBL,  
  xleftmargin=\parindent,
  language=C,
  showstringspaces=false,
  basicstyle=\footnotesize\ttfamily,
  keywordstyle=\bfseries\color{green!40!black},
  commentstyle=\itshape\color{purple!40!black},
  identifierstyle=\color{blue},
  stringstyle=\color{orange},
  tabsize=3
}
\renewcommand{\lstlistingname}{Kodavsnitt}
\lstset{escapechar=@,style=customc}

%----------------- MACROS -----------------
\titleformat{\section}{\fontsize{14}{14}\arialdef\bfseries}{\thesection}{1em}{}
\titleformat{\subsection}{\fontsize{13}{13}\arialdef\bfseries}{\thesubsection}{1em}{}
\titleformat{\subsubsection}{\fontsize{12}{12}\arialdef\bfseries}{\thesubsubsection}{1em}{}

\newcommand{\bilaga}[2]{\subsection*{#1 - #2}}
\newcommand{\tref}[1]{\ref{#1} \textit{\nameref{#1}}}
\newcommand{\kod}[1]{{\ttfamily #1}}
\newcommand{\postbegindocument}
{
	% Fixa framsida här: https://intra.kth.se/kth-cover/
	% Ladda ned PDF och skriv in filnamn nedan.
	\pagenumbering{gobble}
	%\includepdf[pages=1]{pics/kthcover1.pdf}
	\includepdf[pages=1]{pics/\pdf}
	~\clearpage
	
	% Ta bort att författarnamn stryks över i om det är samma författare för olika verk och de hamnar efter varandra i referenser. Ligger även ett kommando högst i bibtex-filen.
	\bstctlcite{IEEEexample:BSTcontrol}
	
	\pagenumbering{roman}	% Romerska siffror för sidonumrering
}

\newcommand{\preenddocument}{~\clearpage \pagenumbering{gobble}~\clearpage
%\includepdf[pages=2]{pics/kthcover1.pdf}
\includepdf[pages=2]{pics/\pdf}
}

\newcommand{\referenser}{\clearpage
\addcontentsline{toc}{section}{Referenser}
\bibliography{referenser.bib}}

\newcommand{\bilagor}{\clearpage
\section*{Bilagor}
\addcontentsline{toc}{section}{Bilagor}
\appendix}


\newenvironment{sammanfattning}{\abstracttitlefont{Sammanfattning}}{}
\renewenvironment{abstract}{\abstracttitlefont{Abstract}}{}
\newenvironment{nyckelord}{~\\~\\\textbf{Nyckelord}\\}{\clearpage}
\newenvironment{keywords}{~\\~\\\textbf{Keywords}\\}{\clearpage}

\newenvironment{definitioner}
{
	\section*{Definitioner}
	\renewcommand{\arraystretch}{1.5}
	\begin{tabular}{l p{12cm}}
}
{
		\end{tabular}
		\renewcommand{\arraystretch}{1}
		\newpage~\clearpage
}
\newdateformat{monthyeardate}{%
  \monthname[\THEMONTH] \THEYEAR}
\newenvironment{foerord}{\section*{Förord}}{\newpage~\newpage\pagenumbering{arabic}}

\newcommand{\figur}[4]
{
	\begin{figure}[h!]
		\centering
		\includegraphics[width=#2\columnwidth]{#1}
		\caption{#3}
		\label{#4}
	\end{figure}
}


% För subsubsubsections
\setcounter{secnumdepth}{4}
% \setcounter{tocdepth}{4}	% Avkommentera för att ha med subsubsubsubsections i ToCen
\titleformat{\paragraph}
{\fontsize{12}{12}\arialit}{\theparagraph}{1em}{}
\titlespacing*{\paragraph}
{0pt}{3.25ex plus 1ex minus .2ex}{1.5ex plus .2ex}
\newcommand{\subsubsubsection}[1]{\paragraph{#1}}

% ---- FÖR TOCEN ---
\addto\captionsswedish{% Replace "english" with the language you use
  \renewcommand{\contentsname}%
    {\fontsize{16}{16}\arialdef\bfseries Innehåll}%
}

% Figurförteckning/tabellförteckning
\renewcommand\cftloftitlefont{\fontsize{16}{16}\arialdef\bfseries}
\renewcommand\cftlottitlefont{\fontsize{16}{16}\arialdef\bfseries}
\addto\captionsswedish{\renewcommand{\listfigurename}{Figurförteckning}}
\addto\captionsswedish{\renewcommand{\listtablename}{Tabellförteckning}}

\renewcommand{\cftsecfont}{\fontsize{12}{12}\arialdef\bfseries}  %Sections text
\renewcommand{\cftsecpagefont}{\fontsize{12}{12}\arialdef\bfseries}  % Sections nummer

\renewcommand{\cftsubsecfont}{\fontsize{11}{11}\arialdef\bfseries}  % Subsections text
\renewcommand{\cftsubsecpagefont}{\fontsize{11}{11}\arialdef\bfseries}  % Subsections nummer

\renewcommand{\cftsubsubsecfont}{\fontsize{11}{11}\arialdef}  % Subsubsections text
\renewcommand{\cftsubsubsecpagefont}{\fontsize{11}{11}\arialdef}  % Subsubsections nummer

\renewcommand{\cftparafont}{\fontsize{11}{11}\arialit}  % Subsubsubsections text
\renewcommand{\cftparapagefont}{\fontsize{11}{11}\arialdef}  % Subsubsubsections nummer

\def\cftdotsep{1}	%	Distances between dots in ToC
\setlength{\cftbeforesecskip}{14pt}	%Add/remove some space before sections in ToC
\setlength{\cftbeforesubsecskip}{5pt}	%Add/remove some space before subsections in ToC
\setlength{\cftbeforesubsubsecskip}{5pt}	%Add/remove some space before subsubsections in ToC


% ---- PAGE NUMBERING IN BOTTOMRIGHT---
\pagestyle{fancy}
\fancyhf{}
\renewcommand{\headrulewidth}{0pt}	% Remove header line
\fancyfoot[LE,RO]{\thepage} % Alternate left and right side

%----------------- STYLES -----------------
\newcommand{\abstracttitlefont}[1]
{\begin{flushleft}\fontsize{16pt}{16pt}\arialbf{#1}\end{flushleft}}

\newcommand{\arialdef}{\normalfont\sffamily}
\newcommand{\arialit}{\itshape\sffamily}

\newcommand{\arial}[1]{{\normalfont\sffamily #1}}
\newcommand{\arialbf}[1]{{\normalfont\sffamily \textbf{#1}}}
% Detta inkluderar koden från denna fil (header.tex).

\documentclass[11pt,a4paper,twoside]{article}

%----------------- PACKAGES -----------------
\usepackage[swedish]{babel}			% Avstavning, språk etc: SWÄRJE
\usepackage[hidelinks]{hyperref}	% Clickable references, links etc.
\usepackage[left=32mm, right=32mm, top=32mm, bottom=32mm]{geometry}		% Page margins
\usepackage[utf8]{inputenc}	% UTF-8 encoding (good for e.g. åäö) 
\usepackage[T1]{fontenc}	% T1 encoding
\usepackage{graphicx}	% For figures
\usepackage{wrapfig}	% For text-wrapped figures
\usepackage{titling}	% For title info etc.
\usepackage{titlesec}	% Formatting of sections etc.
\usepackage{helvet}		% Helvetica, Arial:ish
\usepackage{tocloft}	% ToC fixes
\usepackage{fancyhdr}	% For header and footer formatting
\usepackage{tikz}		% Life sucks without TikZ
\usepackage{circuitikz}	% Life sucks even harder without circuitz
\usetikzlibrary{backgrounds}	% For doing tikz on top of pics
\usepackage{pgfplots}	% For doing plots and stuff
\usepackage{pdfpages}	% For importing cover pages
\usepackage{amsmath}	% Mathematical symbols
\usepackage{gensymb}	% More math symbols
\usepackage[font={footnotesize}]{subcaption}	% For subfigures
\usepackage[font={footnotesize}]{caption}		% Reduce caption font size
\usepackage{float}		% For fancy float options
\usepackage{graphics}	% For placing graphics in /pics/ folder
\usepackage{xcolor}		% Fancy colors
\usepackage{IEEEtrantools}	% For IEEE style formatting
\usepackage{verbatim} % For comment macro
\usepackage{microtype} % Squeeze text to fit margins
\usepackage{listings}	% Listings for code snippets etc.
\usepackage{xcolor}	% Color package
\usepackage{color}	% More colors
\usepackage{setspace}	% Adjust spacings
\usepackage{datetime} % this month etc

%\author{\firstauthorname\\ \secondauthorname}	

%----------------- SETTINGS -----------------
\setlength\parindent{0pt}	% No paragraph indents.
\graphicspath{{./pics/}}		% Put figures in /pics/ folder

\raggedbottom	% Solves problem where section spacing is wacky
\setstretch{1.0}

\bibliographystyle{IEEEtran}


%----------------- C-kod -----------------
\lstdefinestyle{customc}{
  belowcaptionskip=1\baselineskip,
  breaklines=true,
  captionpos=b,
  frame=trBL,  
  xleftmargin=\parindent,
  language=C,
  showstringspaces=false,
  basicstyle=\footnotesize\ttfamily,
  keywordstyle=\bfseries\color{green!40!black},
  commentstyle=\itshape\color{purple!40!black},
  identifierstyle=\color{blue},
  stringstyle=\color{orange},
  tabsize=3
}
\renewcommand{\lstlistingname}{Kodavsnitt}
\lstset{escapechar=@,style=customc}

%----------------- MACROS -----------------
\titleformat{\section}{\fontsize{14}{14}\arialdef\bfseries}{\thesection}{1em}{}
\titleformat{\subsection}{\fontsize{13}{13}\arialdef\bfseries}{\thesubsection}{1em}{}
\titleformat{\subsubsection}{\fontsize{12}{12}\arialdef\bfseries}{\thesubsubsection}{1em}{}

\newcommand{\bilaga}[2]{\subsection*{#1 - #2}}
\newcommand{\tref}[1]{\ref{#1} \textit{\nameref{#1}}}
\newcommand{\kod}[1]{{\ttfamily #1}}
\newcommand{\postbegindocument}
{
	% Fixa framsida här: https://intra.kth.se/kth-cover/
	% Ladda ned PDF och skriv in filnamn nedan.
	\pagenumbering{gobble}
	%\includepdf[pages=1]{pics/kthcover1.pdf}
	\includepdf[pages=1]{pics/\pdf}
	~\clearpage
	
	% Ta bort att författarnamn stryks över i om det är samma författare för olika verk och de hamnar efter varandra i referenser. Ligger även ett kommando högst i bibtex-filen.
	\bstctlcite{IEEEexample:BSTcontrol}
	
	\pagenumbering{roman}	% Romerska siffror för sidonumrering
}

\newcommand{\preenddocument}{~\clearpage \pagenumbering{gobble}~\clearpage
%\includepdf[pages=2]{pics/kthcover1.pdf}
\includepdf[pages=2]{pics/\pdf}
}

\newcommand{\referenser}{\clearpage
\addcontentsline{toc}{section}{Referenser}
\bibliography{referenser.bib}}

\newcommand{\bilagor}{\clearpage
\section*{Bilagor}
\addcontentsline{toc}{section}{Bilagor}
\appendix}


\newenvironment{sammanfattning}{\abstracttitlefont{Sammanfattning}}{}
\renewenvironment{abstract}{\abstracttitlefont{Abstract}}{}
\newenvironment{nyckelord}{~\\~\\\textbf{Nyckelord}\\}{\clearpage}
\newenvironment{keywords}{~\\~\\\textbf{Keywords}\\}{\clearpage}

\newenvironment{definitioner}
{
	\section*{Definitioner}
	\renewcommand{\arraystretch}{1.5}
	\begin{tabular}{l p{12cm}}
}
{
		\end{tabular}
		\renewcommand{\arraystretch}{1}
		\newpage~\clearpage
}
\newdateformat{monthyeardate}{%
  \monthname[\THEMONTH] \THEYEAR}
\newenvironment{foerord}{\section*{Förord}}{\newpage~\newpage\pagenumbering{arabic}}

\newcommand{\figur}[4]
{
	\begin{figure}[h!]
		\centering
		\includegraphics[width=#2\columnwidth]{#1}
		\caption{#3}
		\label{#4}
	\end{figure}
}


% För subsubsubsections
\setcounter{secnumdepth}{4}
% \setcounter{tocdepth}{4}	% Avkommentera för att ha med subsubsubsubsections i ToCen
\titleformat{\paragraph}
{\fontsize{12}{12}\arialit}{\theparagraph}{1em}{}
\titlespacing*{\paragraph}
{0pt}{3.25ex plus 1ex minus .2ex}{1.5ex plus .2ex}
\newcommand{\subsubsubsection}[1]{\paragraph{#1}}

% ---- FÖR TOCEN ---
\addto\captionsswedish{% Replace "english" with the language you use
  \renewcommand{\contentsname}%
    {\fontsize{16}{16}\arialdef\bfseries Innehåll}%
}

% Figurförteckning/tabellförteckning
\renewcommand\cftloftitlefont{\fontsize{16}{16}\arialdef\bfseries}
\renewcommand\cftlottitlefont{\fontsize{16}{16}\arialdef\bfseries}
\addto\captionsswedish{\renewcommand{\listfigurename}{Figurförteckning}}
\addto\captionsswedish{\renewcommand{\listtablename}{Tabellförteckning}}

\renewcommand{\cftsecfont}{\fontsize{12}{12}\arialdef\bfseries}  %Sections text
\renewcommand{\cftsecpagefont}{\fontsize{12}{12}\arialdef\bfseries}  % Sections nummer

\renewcommand{\cftsubsecfont}{\fontsize{11}{11}\arialdef\bfseries}  % Subsections text
\renewcommand{\cftsubsecpagefont}{\fontsize{11}{11}\arialdef\bfseries}  % Subsections nummer

\renewcommand{\cftsubsubsecfont}{\fontsize{11}{11}\arialdef}  % Subsubsections text
\renewcommand{\cftsubsubsecpagefont}{\fontsize{11}{11}\arialdef}  % Subsubsections nummer

\renewcommand{\cftparafont}{\fontsize{11}{11}\arialit}  % Subsubsubsections text
\renewcommand{\cftparapagefont}{\fontsize{11}{11}\arialdef}  % Subsubsubsections nummer

\def\cftdotsep{1}	%	Distances between dots in ToC
\setlength{\cftbeforesecskip}{14pt}	%Add/remove some space before sections in ToC
\setlength{\cftbeforesubsecskip}{5pt}	%Add/remove some space before subsections in ToC
\setlength{\cftbeforesubsubsecskip}{5pt}	%Add/remove some space before subsubsections in ToC


% ---- PAGE NUMBERING IN BOTTOMRIGHT---
\pagestyle{fancy}
\fancyhf{}
\renewcommand{\headrulewidth}{0pt}	% Remove header line
\fancyfoot[LE,RO]{\thepage} % Alternate left and right side

%----------------- STYLES -----------------
\newcommand{\abstracttitlefont}[1]
{\begin{flushleft}\fontsize{16pt}{16pt}\arialbf{#1}\end{flushleft}}

\newcommand{\arialdef}{\normalfont\sffamily}
\newcommand{\arialit}{\itshape\sffamily}

\newcommand{\arial}[1]{{\normalfont\sffamily #1}}
\newcommand{\arialbf}[1]{{\normalfont\sffamily \textbf{#1}}}
% Detta inkluderar koden från denna fil (header.tex).

\documentclass[11pt,a4paper,twoside]{article}

%----------------- PACKAGES -----------------
\usepackage[swedish]{babel}			% Avstavning, språk etc: SWÄRJE
\usepackage[hidelinks]{hyperref}	% Clickable references, links etc.
\usepackage[left=32mm, right=32mm, top=32mm, bottom=32mm]{geometry}		% Page margins
\usepackage[utf8]{inputenc}	% UTF-8 encoding (good for e.g. åäö) 
\usepackage[T1]{fontenc}	% T1 encoding
\usepackage{graphicx}	% For figures
\usepackage{wrapfig}	% For text-wrapped figures
\usepackage{titling}	% For title info etc.
\usepackage{titlesec}	% Formatting of sections etc.
\usepackage{helvet}		% Helvetica, Arial:ish
\usepackage{tocloft}	% ToC fixes
\usepackage{fancyhdr}	% For header and footer formatting
\usepackage{tikz}		% Life sucks without TikZ
\usepackage{circuitikz}	% Life sucks even harder without circuitz
\usetikzlibrary{backgrounds}	% For doing tikz on top of pics
\usepackage{pgfplots}	% For doing plots and stuff
\usepackage{pdfpages}	% For importing cover pages
\usepackage{amsmath}	% Mathematical symbols
\usepackage{gensymb}	% More math symbols
\usepackage[font={footnotesize}]{subcaption}	% For subfigures
\usepackage[font={footnotesize}]{caption}		% Reduce caption font size
\usepackage{float}		% For fancy float options
\usepackage{graphics}	% For placing graphics in /pics/ folder
\usepackage{xcolor}		% Fancy colors
\usepackage{IEEEtrantools}	% For IEEE style formatting
\usepackage{verbatim} % For comment macro
\usepackage{microtype} % Squeeze text to fit margins
\usepackage{listings}	% Listings for code snippets etc.
\usepackage{xcolor}	% Color package
\usepackage{color}	% More colors
\usepackage{setspace}	% Adjust spacings
\usepackage{datetime} % this month etc

%\author{\firstauthorname\\ \secondauthorname}	

%----------------- SETTINGS -----------------
\setlength\parindent{0pt}	% No paragraph indents.
\graphicspath{{./pics/}}		% Put figures in /pics/ folder

\raggedbottom	% Solves problem where section spacing is wacky
\setstretch{1.0}

\bibliographystyle{IEEEtran}


%----------------- C-kod -----------------
\lstdefinestyle{customc}{
  belowcaptionskip=1\baselineskip,
  breaklines=true,
  captionpos=b,
  frame=trBL,  
  xleftmargin=\parindent,
  language=C,
  showstringspaces=false,
  basicstyle=\footnotesize\ttfamily,
  keywordstyle=\bfseries\color{green!40!black},
  commentstyle=\itshape\color{purple!40!black},
  identifierstyle=\color{blue},
  stringstyle=\color{orange},
  tabsize=3
}
\renewcommand{\lstlistingname}{Kodavsnitt}
\lstset{escapechar=@,style=customc}

%----------------- MACROS -----------------
\titleformat{\section}{\fontsize{14}{14}\arialdef\bfseries}{\thesection}{1em}{}
\titleformat{\subsection}{\fontsize{13}{13}\arialdef\bfseries}{\thesubsection}{1em}{}
\titleformat{\subsubsection}{\fontsize{12}{12}\arialdef\bfseries}{\thesubsubsection}{1em}{}

\newcommand{\bilaga}[2]{\subsection*{#1 - #2}}
\newcommand{\tref}[1]{\ref{#1} \textit{\nameref{#1}}}
\newcommand{\kod}[1]{{\ttfamily #1}}
\newcommand{\postbegindocument}
{
	% Fixa framsida här: https://intra.kth.se/kth-cover/
	% Ladda ned PDF och skriv in filnamn nedan.
	\pagenumbering{gobble}
	%\includepdf[pages=1]{pics/kthcover1.pdf}
	\includepdf[pages=1]{pics/\pdf}
	~\clearpage
	
	% Ta bort att författarnamn stryks över i om det är samma författare för olika verk och de hamnar efter varandra i referenser. Ligger även ett kommando högst i bibtex-filen.
	\bstctlcite{IEEEexample:BSTcontrol}
	
	\pagenumbering{roman}	% Romerska siffror för sidonumrering
}

\newcommand{\preenddocument}{~\clearpage \pagenumbering{gobble}~\clearpage
%\includepdf[pages=2]{pics/kthcover1.pdf}
\includepdf[pages=2]{pics/\pdf}
}

\newcommand{\referenser}{\clearpage
\addcontentsline{toc}{section}{Referenser}
\bibliography{referenser.bib}}

\newcommand{\bilagor}{\clearpage
\section*{Bilagor}
\addcontentsline{toc}{section}{Bilagor}
\appendix}


\newenvironment{sammanfattning}{\abstracttitlefont{Sammanfattning}}{}
\renewenvironment{abstract}{\abstracttitlefont{Abstract}}{}
\newenvironment{nyckelord}{~\\~\\\textbf{Nyckelord}\\}{\clearpage}
\newenvironment{keywords}{~\\~\\\textbf{Keywords}\\}{\clearpage}

\newenvironment{definitioner}
{
	\section*{Definitioner}
	\renewcommand{\arraystretch}{1.5}
	\begin{tabular}{l p{12cm}}
}
{
		\end{tabular}
		\renewcommand{\arraystretch}{1}
		\newpage~\clearpage
}
\newdateformat{monthyeardate}{%
  \monthname[\THEMONTH] \THEYEAR}
\newenvironment{foerord}{\section*{Förord}}{\newpage~\newpage\pagenumbering{arabic}}

\newcommand{\figur}[4]
{
	\begin{figure}[h!]
		\centering
		\includegraphics[width=#2\columnwidth]{#1}
		\caption{#3}
		\label{#4}
	\end{figure}
}


% För subsubsubsections
\setcounter{secnumdepth}{4}
% \setcounter{tocdepth}{4}	% Avkommentera för att ha med subsubsubsubsections i ToCen
\titleformat{\paragraph}
{\fontsize{12}{12}\arialit}{\theparagraph}{1em}{}
\titlespacing*{\paragraph}
{0pt}{3.25ex plus 1ex minus .2ex}{1.5ex plus .2ex}
\newcommand{\subsubsubsection}[1]{\paragraph{#1}}

% ---- FÖR TOCEN ---
\addto\captionsswedish{% Replace "english" with the language you use
  \renewcommand{\contentsname}%
    {\fontsize{16}{16}\arialdef\bfseries Innehåll}%
}

% Figurförteckning/tabellförteckning
\renewcommand\cftloftitlefont{\fontsize{16}{16}\arialdef\bfseries}
\renewcommand\cftlottitlefont{\fontsize{16}{16}\arialdef\bfseries}
\addto\captionsswedish{\renewcommand{\listfigurename}{Figurförteckning}}
\addto\captionsswedish{\renewcommand{\listtablename}{Tabellförteckning}}

\renewcommand{\cftsecfont}{\fontsize{12}{12}\arialdef\bfseries}  %Sections text
\renewcommand{\cftsecpagefont}{\fontsize{12}{12}\arialdef\bfseries}  % Sections nummer

\renewcommand{\cftsubsecfont}{\fontsize{11}{11}\arialdef\bfseries}  % Subsections text
\renewcommand{\cftsubsecpagefont}{\fontsize{11}{11}\arialdef\bfseries}  % Subsections nummer

\renewcommand{\cftsubsubsecfont}{\fontsize{11}{11}\arialdef}  % Subsubsections text
\renewcommand{\cftsubsubsecpagefont}{\fontsize{11}{11}\arialdef}  % Subsubsections nummer

\renewcommand{\cftparafont}{\fontsize{11}{11}\arialit}  % Subsubsubsections text
\renewcommand{\cftparapagefont}{\fontsize{11}{11}\arialdef}  % Subsubsubsections nummer

\def\cftdotsep{1}	%	Distances between dots in ToC
\setlength{\cftbeforesecskip}{14pt}	%Add/remove some space before sections in ToC
\setlength{\cftbeforesubsecskip}{5pt}	%Add/remove some space before subsections in ToC
\setlength{\cftbeforesubsubsecskip}{5pt}	%Add/remove some space before subsubsections in ToC


% ---- PAGE NUMBERING IN BOTTOMRIGHT---
\pagestyle{fancy}
\fancyhf{}
\renewcommand{\headrulewidth}{0pt}	% Remove header line
\fancyfoot[LE,RO]{\thepage} % Alternate left and right side

%----------------- STYLES -----------------
\newcommand{\abstracttitlefont}[1]
{\begin{flushleft}\fontsize{16pt}{16pt}\arialbf{#1}\end{flushleft}}

\newcommand{\arialdef}{\normalfont\sffamily}
\newcommand{\arialit}{\itshape\sffamily}

\newcommand{\arial}[1]{{\normalfont\sffamily #1}}
\newcommand{\arialbf}[1]{{\normalfont\sffamily \textbf{#1}}}
% Detta inkluderar koden från denna fil (header.tex).

\documentclass[11pt,a4paper,twoside]{article}

%----------------- PACKAGES -----------------
\usepackage[swedish]{babel}			% Avstavning, språk etc: SWÄRJE
\usepackage[hidelinks]{hyperref}	% Clickable references, links etc.
\usepackage[left=32mm, right=32mm, top=32mm, bottom=32mm]{geometry}		% Page margins
\usepackage[utf8]{inputenc}	% UTF-8 encoding (good for e.g. åäö) 
\usepackage[T1]{fontenc}	% T1 encoding
\usepackage{graphicx}	% For figures
\usepackage{wrapfig}	% For text-wrapped figures
\usepackage{titling}	% For title info etc.
\usepackage{titlesec}	% Formatting of sections etc.
\usepackage{helvet}		% Helvetica, Arial:ish
\usepackage{tocloft}	% ToC fixes
\usepackage{fancyhdr}	% For header and footer formatting
\usepackage{tikz}		% Life sucks without TikZ
\usepackage{circuitikz}	% Life sucks even harder without circuitz
\usetikzlibrary{backgrounds}	% For doing tikz on top of pics
\usepackage{pgfplots}	% For doing plots and stuff
\usepackage{pdfpages}	% For importing cover pages
\usepackage{amsmath}	% Mathematical symbols
\usepackage{gensymb}	% More math symbols
\usepackage[font={footnotesize}]{subcaption}	% For subfigures
\usepackage[font={footnotesize}]{caption}		% Reduce caption font size
\usepackage{float}		% For fancy float options
\usepackage{graphics}	% For placing graphics in /pics/ folder
\usepackage{xcolor}		% Fancy colors
\usepackage{IEEEtrantools}	% For IEEE style formatting
\usepackage{verbatim} % For comment macro
\usepackage{microtype} % Squeeze text to fit margins
\usepackage{listings}	% Listings for code snippets etc.
\usepackage{xcolor}	% Color package
\usepackage{color}	% More colors
\usepackage{setspace}	% Adjust spacings
\usepackage{datetime} % this month etc

\author{\authorname}	

%----------------- SETTINGS -----------------
\setlength\parindent{0pt}	% No paragraph indents.
\graphicspath{{./pics/}}		% Put figures in /pics/ folder

\raggedbottom	% Solves problem where section spacing is wacky
\setstretch{1.0}

\bibliographystyle{IEEEtran}


%----------------- C-kod -----------------
\lstdefinestyle{customc}{
  belowcaptionskip=1\baselineskip,
  breaklines=true,
  captionpos=b,
  frame=trBL,  
  xleftmargin=\parindent,
  language=C,
  showstringspaces=false,
  basicstyle=\footnotesize\ttfamily,
  keywordstyle=\bfseries\color{green!40!black},
  commentstyle=\itshape\color{purple!40!black},
  identifierstyle=\color{blue},
  stringstyle=\color{orange},
  tabsize=3
}
\renewcommand{\lstlistingname}{Kodavsnitt}
\lstset{escapechar=@,style=customc}

%----------------- MACROS -----------------
\titleformat{\section}{\fontsize{14}{14}\arialdef\bfseries}{\thesection}{1em}{}
\titleformat{\subsection}{\fontsize{13}{13}\arialdef\bfseries}{\thesubsection}{1em}{}
\titleformat{\subsubsection}{\fontsize{12}{12}\arialdef\bfseries}{\thesubsubsection}{1em}{}

\newcommand{\bilaga}[2]{\subsection*{#1 - #2}}
\newcommand{\tref}[1]{\ref{#1} \textit{\nameref{#1}}}
\newcommand{\kod}[1]{{\ttfamily #1}}
\newcommand{\postbegindocument}
{
	% Fixa framsida här: https://intra.kth.se/kth-cover/
	% Ladda ned PDF och skriv in filnamn nedan.
	\pagenumbering{gobble}
	%\includepdf[pages=1]{pics/kthcover1.pdf}
	\includepdf[pages=1]{pics/\pdf}
	~\clearpage
	
	% Ta bort att författarnamn stryks över i om det är samma författare för olika verk och de hamnar efter varandra i referenser. Ligger även ett kommando högst i bibtex-filen.
	\bstctlcite{IEEEexample:BSTcontrol}
	
	\pagenumbering{roman}	% Romerska siffror för sidonumrering
}

\newcommand{\preenddocument}{~\clearpage \pagenumbering{gobble}~\clearpage
%\includepdf[pages=2]{pics/kthcover1.pdf}
\includepdf[pages=2]{pics/\pdf}
}

\newcommand{\referenser}{\clearpage
\addcontentsline{toc}{section}{Referenser}
\bibliography{referenser.bib}}

\newcommand{\bilagor}{\clearpage
\section*{Bilagor}
\addcontentsline{toc}{section}{Bilagor}
\appendix}


\newenvironment{sammanfattning}{\abstracttitlefont{Sammanfattning}}{}
\renewenvironment{abstract}{\abstracttitlefont{Abstract}}{}
\newenvironment{nyckelord}{~\\~\\\textbf{Nyckelord}\\}{\clearpage}
\newenvironment{keywords}{~\\~\\\textbf{Keywords}\\}{\clearpage}

\newenvironment{definitioner}
{
	\section*{Definitioner}
	\renewcommand{\arraystretch}{1.5}
	\begin{tabular}{l p{12cm}}
}
{
		\end{tabular}
		\renewcommand{\arraystretch}{1}
		\newpage~\clearpage
}
\newdateformat{monthyeardate}{%
  \monthname[\THEMONTH] \THEYEAR}
\newenvironment{foerord}{\section*{Förord}}{~\\[2cm] Stockholm, \monthyeardate\today\\\theauthor
\newpage~\newpage\pagenumbering{arabic}}

\newcommand{\figur}[4]
{
	\begin{figure}[h!]
		\centering
		\includegraphics[width=#2\columnwidth]{#1}
		\caption{#3}
		\label{#4}
	\end{figure}
}


% För subsubsubsections
\setcounter{secnumdepth}{4}
% \setcounter{tocdepth}{4}	% Avkommentera för att ha med subsubsubsubsections i ToCen
\titleformat{\paragraph}
{\fontsize{12}{12}\arialit}{\theparagraph}{1em}{}
\titlespacing*{\paragraph}
{0pt}{3.25ex plus 1ex minus .2ex}{1.5ex plus .2ex}
\newcommand{\subsubsubsection}[1]{\paragraph{#1}}

% ---- FÖR TOCEN ---
\addto\captionsswedish{% Replace "english" with the language you use
  \renewcommand{\contentsname}%
    {\fontsize{16}{16}\arialdef\bfseries Innehåll}%
}

% Figurförteckning/tabellförteckning
\renewcommand\cftloftitlefont{\fontsize{16}{16}\arialdef\bfseries}
\renewcommand\cftlottitlefont{\fontsize{16}{16}\arialdef\bfseries}
\addto\captionsswedish{\renewcommand{\listfigurename}{Figurförteckning}}
\addto\captionsswedish{\renewcommand{\listtablename}{Tabellförteckning}}

\renewcommand{\cftsecfont}{\fontsize{12}{12}\arialdef\bfseries}  %Sections text
\renewcommand{\cftsecpagefont}{\fontsize{12}{12}\arialdef\bfseries}  % Sections nummer

\renewcommand{\cftsubsecfont}{\fontsize{11}{11}\arialdef\bfseries}  % Subsections text
\renewcommand{\cftsubsecpagefont}{\fontsize{11}{11}\arialdef\bfseries}  % Subsections nummer

\renewcommand{\cftsubsubsecfont}{\fontsize{11}{11}\arialdef}  % Subsubsections text
\renewcommand{\cftsubsubsecpagefont}{\fontsize{11}{11}\arialdef}  % Subsubsections nummer

\renewcommand{\cftparafont}{\fontsize{11}{11}\arialit}  % Subsubsubsections text
\renewcommand{\cftparapagefont}{\fontsize{11}{11}\arialdef}  % Subsubsubsections nummer

\def\cftdotsep{1}	%	Distances between dots in ToC
\setlength{\cftbeforesecskip}{14pt}	%Add/remove some space before sections in ToC
\setlength{\cftbeforesubsecskip}{5pt}	%Add/remove some space before subsections in ToC
\setlength{\cftbeforesubsubsecskip}{5pt}	%Add/remove some space before subsubsections in ToC


% ---- PAGE NUMBERING IN BOTTOMRIGHT---
\pagestyle{fancy}
\fancyhf{}
\renewcommand{\headrulewidth}{0pt}	% Remove header line
\fancyfoot[LE,RO]{\thepage} % Alternate left and right side

%----------------- STYLES -----------------
\newcommand{\abstracttitlefont}[1]
{\begin{flushleft}\fontsize{16pt}{16pt}\arialbf{#1}\end{flushleft}}

\newcommand{\arialdef}{\normalfont\sffamily}
\newcommand{\arialit}{\itshape\sffamily}

\newcommand{\arial}[1]{{\normalfont\sffamily #1}}
\newcommand{\arialbf}[1]{{\normalfont\sffamily \textbf{#1}}}