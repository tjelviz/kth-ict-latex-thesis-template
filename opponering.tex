\documentclass[11pt,a4paper]{article}

\usepackage[hidelinks]{hyperref}
\usepackage[left=32mm, right=32mm, top=32mm, bottom=32mm]{geometry}
\usepackage[utf8]{inputenc}
\usepackage[T1]{fontenc}
\usepackage{helvet}
\usepackage{microtype}
\usepackage{titlesec}
\usepackage{amsmath}
\usepackage{amsfonts}
\usepackage[swedish]{babel}

%------- FONTS ------
\titleformat{\section}
{\fontsize{14}{14}\arialdef\bfseries}{\thesection}{1em}{}

\titleformat{\subsection}
{\fontsize{13}{13}\arialdef\bfseries}{\thesubsection}{1em}{}

\titleformat{\subsubsection}
{\fontsize{11}{11}\arialdef\bfseries}{\thesubsubsection}{1em}{}

\newcommand{\arialdef}{\normalfont\sffamily}
\newcommand{\arialit}{\itshape\sffamily}
\newcommand{\email}[1]{\href{mailto:#1}{#1}}
\setlength\parindent{0pt}
\raggedbottom

\hyphenation{quad-kopter-modellen tids-konstant}

\begin{document}
\section*{Opposition on written report / thesis\hfill\footnotesize\today}
The document length is about 2 pages. Hand in the opposition \underline{\textit{at least}} 1 day before seminar! In the opposition, write the title of the report that the opposition concerns and the name of the authors who wrote the report and email information. Also write your name as reviewing 
student, see:
\subsection*{Report/thesis}
<Title of report>\\
<Name of author(s)>\\
\email{author@kth.se}
\subsection*{Reviewing student}
<Name of reviewer/opponent>\\
\email{opponent@kth.se}

\section*{Contents of the opposition}
The purpose of opposition is to constructively criticize a student’s material. Think of the implementation of the opposition in a positive spirit. Start by talking about what you think is 
good for the work carried out. The goal of opposition is help, not just judge the work. This 
means that you should give positive and negative feedback. Name at least three positive 
things with the work. The negative feedback should include comments about possible ways to 
improve the work. Give an example of improvement for each comment. Below you find suggestions of contents that might be in opposition - note that this material should only be used as suggestions for content! The following points may be appropriate to discuss:

\subsection*{Overall assessment}
Do you see a logical presentation in the report, so call
ed “red thread”?\\
Are the individual sections logically connected to each other?\\
Is the thesis understandable? Is something missing?\\
Is the thesis and its content consistent and overall trustable?( evaluated, verified, tested) 

\subsection*{Abstract}
Does it present the background, problem \& purposes, what has been studied, results of the 
study and conclusions. Is the abstract understandable by its own?

\subsection*{introduction}
Does the introduction give the big picture of the topic?\\
Does it have relevant references?

\subsection*{Subject and problem area}
Is the chosen topic and problem area of interest?\\
Does the topic have a clear introduction, theory and background?\\
How are the theories chosen?\\
Is there a relevant theory that has not been included in the report?\\
How are theories described?

\subsection*{Problem discussion and aim}
Do the authors manage to attract the reader to the area?\\
Do they motivate the choice of topic?\\
Is it clear what problem statements the authors had from the beginning?\\
Do you think any important issues have been overlooked?\\
Does the aim of the report specify the problem discussion? \\
Is the purpose clearly described?\\
Are the goals, objectives, and/or deliverables achievable?

\subsection*{Boundaries}
Does the work have reasonable boundaries? \\
Does delimitation connect to the thesis?\\
Are the boundaries justified?

\subsection*{Disposition}
Does it guide through the material? Is it well-balanced? Does it cover all parts?

\subsection*{Method}
Based the problem statements, have the authors made a conscious choice of method\\
Is the chosen study approach (e.g. engineering (CDI(T)O – Conceive, Design, Implement, Test, Operate), case study, survey, action-research or experiments and planned trials), appropriate?\\ 
Is the choice of method properly motivated? Has the method been a conscious choice?\\
Are there feasible alternatives to the chosen method? Are the methods described and discussed? Well referenced?\\
Do the authors discuss validity, reliability, replication, or dependability, ethics and sustainability? Are all these parts needed for the investigation?

\subsection*{Data collection}
How did the author perform the data collection? Do the authors discuss the methods (for data 
collection) conformity with the method?\\
Is the empirical description rigorous and logical in the light of the purpose?\\
Is the empirical description of an appropriate scale?

\subsection*{Interpreting material (Data analysis)}
How have the authors made use of their empirical material?\\
Do they use the learning described in the frame of reference for interpreting data?\\
How does the analysis match the content of the report?

\subsection*{Results}
Are the findings and conclusions sustainable?  Are they based on the analysis of data?\\
Are suggestions or recommendations given to the reader or clients (if clients exist)?\\
Are the goals, objective, and deliverables achieved?\\
If testing is used, are test results used as proof (are they convincing)? 

\subsection*{Conclusions and discussion}
Is the purpose, goal mentioned in the Introduction achieved? Is the problem solved (validity)? \\
How well the problem is solved (reliability)? \\
Is there any evaluation of the methods that have been used?\\
Is the author(s) true to the data?\\
Are the limitations in the study discussed?\\
Are there specified proposals for future work?\\
Is the outcome of the project evaluated by third party?\\ 
Are there relevant comments and considerations on ethics and sustainability?

\subsection*{References}
What are reference did the authors chose to use? Are the references useful?\\
Are the references presented in a consistent manner?\\
Are the references valid? Are the references justified (age, topic and content)?

\subsection*{Language and Technical Performance}
Does the report contain typo errors, incorrect sentence structures or other similar defects?\\
Is the report divided into logical pieces?\\
Are the table of contents, headings and references to sources of consistent and accurate?\\
Are figures and tables done correctly?

\subsection*{Honesty and Critical Distance}
Is it easy for the reader to distinguish what is taken from the literature and other sources and what the authors’ own opinions?\\
Do the authors show a critical distance to the theories and conclusions?

\subsection*{General Impression}
Do the authors contribute something new in the chosen field?

\subsection*{Other Issues}
Is the material coherent? Are the titles of the sections correct? Are the figures (tables) well selected and illustrated? Are the figures/tables described in the text? Do the figures (tables) have a clear purpose and fulfil the purpose?

\subsection*{Appendices}
List the specific details that should be corrected that do not. (All the things you cannot declare in your oral opposition but are of value for the respondent in order to improve the report)

\end{document}